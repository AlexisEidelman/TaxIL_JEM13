\documentclass[xcolor=table,ignorenonframetext,12pt]{beamer}
%\documentclass[xcolor=table,handout,12pt]{beamer}
\usepackage[frenchb]{babel}
\usepackage[latin1]{inputenc}
\usepackage{amsmath,amssymb}
\usepackage{graphicx}
\usepackage{pgfarrows,pgfnodes}
\usepackage{url}
\usepackage{textcomp}
\usepackage[vcentermath]{youngtab}
\usepackage{epstopdf}
\usepackage{hhline}
\usepackage{xmpmulti}
\usepackage{subfigure}
\usepackage{tikz}

\mode<article> {
  \usepackage{fullpage}
  \usepackage{pgf}
  \usepackage{hyperref}
}
\newcommand{\til}{TaxIPP-Life}

\mode<presentation>

\usetheme[left]{Goettingen}
%\setbeamertemplate{background canvas}{\includegraphics[width=\paperwidth]{ippheader.pdf}}
\setbeamertemplate{background canvas}{\hspace{10.1cm}\includegraphics
    [width=0.2\paperwidth]{logo_ipp.pdf}}
\definecolor{ippdark}{RGB}{0,93,116}
\definecolor{ipplight}{RGB}{0,142,156}
\useinnertheme[shadow=true]{rounded}
\setbeamertemplate{items}[circle]
\setbeamercolor{title}{bg=ipplight, fg=black}
\setbeamercolor{structure}{fg=ipplight}
%\setbeamertemplate{sidebar canvas left}{} % pour supprimer le fond de couleur de la barre lat�rale	
\setbeamertemplate{sidebar canvas left}[vertical shading][top=structure.fg!20,bottom=structure.fg!15]
\setbeamertemplate{caption}[numbered]
\setlength{\leftmargini}{12pt}
\setbeamerfont{framesubtitle}{size=\large}

%\usepackage[latin1]{inputenc}
%\usepackage[english]{babel}

\newenvironment{checklist}[1]{\begin{list}{$\surd$}{}#1}{\end{list}}%
\newcommand{\graphique}[2][1]{\begin{minipage}{\linewidth}\begin{center}\includegraphics[width=#1\linewidth,clip]{#2}\end{center}\end{minipage}}%
\newcommand{\os}[2]{\onslide+<#1->{#2}}%

\newcommand{\m}[2]{\multicolumn{#1}{c}{#2}}%
\newcommand{\ml}[2]{\multicolumn{#1}{l}{#2}}%
\newcommand{\tth}{\textsuperscript{th}}%
\newcommand{\nde}{\textsuperscript{nde}}
\newcommand{\hligne}{\begin{tikzpicture}[remember picture,overlay]\node[shift={(-10.5 cm,-1.2cm)}]at(current page.north east){\begin{tikzpicture}{\draw[line width=0.2mm,color=ipplight,overlay](0,0)--(8,0);}\end{tikzpicture}};\end{tikzpicture}\vspace{-0.8cm}}
\newcommand{\hlignee}{\begin{tikzpicture}[remember picture,overlay]\node[shift={(-10.5 cm,-1.2cm)}]at(current page.north east){\begin{tikzpicture}{\draw[line width=0.2mm,color=ipplight,overlay](0,0)--(8,0);}\end{tikzpicture}};\end{tikzpicture}}

% MTABLE: macro for tables %
\newenvironment{mfigure}[4][1]{\def\TMP{#3}\newdimen\TMPsize\settowidth{\TMPsize}{\TMP}
\begin{figure}\caption{#2}\begin{center}\begin{tiny}
\begin{minipage}{#1\textwidth}\resizebox{\textwidth}{!}{#3}\end{minipage}
\if!#4!\empty \else \\
\resizebox{#1\textwidth}{!}{\begin{minipage}{\TMPsize}\begin{tiny}\smallskip\par
#4 \end{tiny}\end{minipage}} \fi}
{\end{tiny}\end{center}\end{figure}}

\makeatletter
\def\TAXIPP{TAX\kern-.05em\lower-.19ex\hbox{${\color{BlueGreen} \scalebox{1.4}{\underarc[1]{\overarc[1]{\textcolor{black}{\scalebox{0.7}{ipp}}}}}}\hspace{0.5ex}$}\@}
\makeatother

\title{Simuler le cycle de vie\\ Le mod�le \til}
\author{Alexis Eidelman}
\institute{\emph{Institut des politiques publiques}}
\subject{PSE - microsimulation}

\date{Exp�riences et perspectives de la micro-simulation \\Montreuil -- 23 mai 2013}

\begin{document}

\frame{\maketitle}


\section{Introduction}

\begin{frame}
  \hligne
  \frametitle{Introduction}
  \framesubtitle{\vspace{0.3cm} Pourquoi faire des simulations sur le cycle de vie ?}
  \begin{itemize}
    \item  mesure d'un \og meilleur \fg \ niveau de vie
        \begin{itemize}
          \item Variabilit� des revenus
          \item �quit� dans le traitement des individus vis-�-vis de leur �ge. 
          \item Prise en compte du patrimoine
        \end{itemize} \pause
    \item Possibilit� d'arbitrer entre part assurantielle et part redistributive des transferts \pause
    \item Capacit� de pr�diction \pause
    \item ...et bien d'autres \pause
  \end{itemize}
\end{frame}

\begin{frame}
  \hligne
  \frametitle{Introduction}
  \framesubtitle{\vspace{0.3cm}Organisation}\pause
  \begin{itemize}
    \item \textbf{Language}
    \begin{itemize}
    \item Traitement initial des donn�es en R
    \item Utilisation principale de Python
    \end{itemize} 
     \item \textbf{Pourquoi Python} ?\pause
    \begin{itemize}
    	\item La microsimulation n'est pas essentiellement statistique, pourquoi se limiter � un langage de stat ?
    	\item Langage clair. Possibilit� d'avoir plusieurs niveaux de lecture, tr�s utile pour la microsimulation 
   		\item Gratuit�
		\item Performances tr�s int�ressantes (important pour \til  voir plus loin)	
    \end{itemize} 
	Un choix que d'autres ont fait avant moi (voir plus loin)
 )
  \end{itemize}
\end{frame}

\begin{frame}
  \hligne
  \frametitle{Introduction}
  \framesubtitle{\vspace{0.3cm}Organisation}\pause
  \begin{itemize}
    \item \textbf{Language}
    \begin{itemize}
    \item Traitement initial des donn�es en R
    \item Utilisation principale de Python
    \end{itemize} 
     \item \textbf{Pourquoi Python} ?\pause
    \begin{itemize}
    	\item La microsimulation n'est pas essentiellement statistique, pourquoi se limiter � un langage de stat ?
    	\item Langage clair. Possibilit� d'avoir plusieurs niveaux de lecture, tr�s utile pour la microsimulation 
   		\item Gratuit�
		\item Performances tr�s int�ressante	
    \end{itemize} 
	Un choix que d'autres ont fait avant moi (voir plus loin)

  \end{itemize}
\end{frame}




\begin{frame}
  \hligne
  \frametitle{Introduction}
  \framesubtitle{\vspace{0.3cm}Outline}
  \begin{itemize}
    \item[I.] Data setup
    \item[II.] Microsimulation model
    \item[III.] Measuring redistribution
    \item[IV.] Ongoing and future developments
  \end{itemize}
\end{frame}


\section{I. Data}

\begin{frame}
  \hligne
  \frametitle{I. Data}
  \framesubtitle{\vspace{0.3cm}}
  \begin{enumerate}
    \item Sources
    \item Matching procedures
    \item Imputing top incomes
    \item Matching-up on aggregate data
  \end{enumerate}
\end{frame}

\subsection{Sources}

\begin{frame}
  \hligne
  \frametitle{I. Data}
  \framesubtitle{\vspace{0.3cm}Sources}\pause
  \begin{itemize}
    \item \textbf{Micro-data}
    \begin{itemize}
      \item Household survey \textit{Revenus fiscaux} (Labour force survey matched to tax and benefit data)
      \item Household survey \textit{Budget des familles}
      \item Household survey \textit{Patrimoine}
      \item Household survey \textit{Logement}
      \item Administrative tax data on top incomes (Top incomes world database, A. Atkinson and T. Piketty, 2007)
    \end{itemize}\pause
    \item \textbf{Aggregate data}
    \begin{itemize}
    \item Demographics
    \item National accounts
    \item Detailed tax revenues
    \item Benefits administrative data
    \end{itemize}
  \end{itemize}
\end{frame}

\begin{frame}
 \hligne
  \frametitle{I. Data}
  \framesubtitle{\vspace{0.3cm}Example: tax revenues}\pause
\end{frame}


\subsection{Matching}

\begin{frame}
  \hligne
  \frametitle{I. Data}
  \framesubtitle{\vspace{0.3cm}Matching data sources}\pause
  \begin{itemize}
    \item \textbf{Principle}
    \begin{itemize}
      \item Literature on data fusion
      \item Select common variables (age, sex, household composition, income, types of income)
      \item Create a score and minimize distance
    \end{itemize}\pause
    \item \textbf{Practice}
    \begin{itemize}
    \item Main source is \textit{Revenus fiscaux 2006}
    \item Matched with housing data from \textit{Logement 2006}
    \item Matched with wealth data from \textit{Patrimoine 2005}
    \item Matched with consumption data from \textit{Budget des familles 2006}
    \end{itemize}
  \end{itemize}
\end{frame}

\subsection{Top incomes}

\begin{frame}
  \hligne
  \frametitle{I. Data}
  \framesubtitle{\vspace{0.3cm}Imputing top incomes}\pause
  \begin{itemize}
    \item \textbf{Limitation of survey sources}
    \begin{itemize}
      \item Top of the distribution not well represented
      \item Too few observations, under-reporting
      \item Generally explains the under estimation of aggregate values
    \end{itemize}\pause
    \item \textbf{Solution}
    \begin{itemize}
    \item Using tax data from administrative sources
    \item Estimation of top income distribution (Piketty and Saez, 2001; Atkinson and Piketty, 2010)
    \item Impute top incomes based on these distributions
    \end{itemize}
  \end{itemize}
\end{frame}

\subsection{Aggregate data}

\begin{frame}
  \hligne
  \frametitle{I. Data}
  \framesubtitle{\vspace{0.3cm}Weighting-up to aggregate data}\pause
  \begin{itemize}
  \item \textbf{Principle}
    \begin{itemize}
      \item Systematic comparison between aggregates from micro data and macro data
      \item Run the model to get estimates of tax revenues by type of revenues
      \item Re-base variables on macro-data
    \end{itemize} \pause
  \item \textbf{Discrepancies}
    \begin{itemize}
    \item Earnings:
    \begin{itemize}
      \item Good fit for private sector
      \item No identification of bonuses in public sector
      \item 10\% lower estimate than NA (black market, fringe benefits)
    \end{itemize}
    \item Other income: much lower estimate of dividends and other capital income estimates
    \end{itemize}
  \end{itemize}
\end{frame}

\begin{frame}
  \hligne
  \frametitle{I. Data}
  \framesubtitle{\vspace{0.3cm}Weighting-up to aggregate data}\pause
\footnotesize
\begin{table}
\caption{Ratio of simulated gross earnings to aggregate estimates from CSG tax base and national accounts (NA)}
\begin{tabular}{|c|cccc|}
\hline
      & \multicolumn{4}{c|}{\textbf{Gross earnings (simulated) /}}   \\
Year  & \multicolumn{2}{c}{CSG tax base} & \multicolumn{2}{c|}{NA tax base}  \\
      & private & public  & private & public   \\
\hline
 2004 &  99,6\%	&  98,9\% &	91,9\% & 89,9\% \\
 2005 &  99,4\%	&  99,2\% &	90,9\% & 90,9\% \\
 2006 &  99,4\%	& 101,4\% &	90,8\% & 92,4\% \\
 2007 &  99,5\%	& 103,0\% &	90,5\% & 93,6\% \\
 2008 &  98,5\%	& 105,7\% &	91,0\% & 95,5\% \\
 2009 &  97,1\%	& 102,6\% &	91,8\% & 93,0\% \\
 2010 & 100,7\%	& 104,1\% &	91,8\% & 93,0\% \\
\hline
\end{tabular}
\end{table}
\end{frame}

\section{II. Model}

\begin{frame}
  \hligne
  \frametitle{II. Model}
  \framesubtitle{\vspace{0.3cm}}
  \begin{enumerate}
    \item Simulating legislation
    \item Incidence assumptions
    \item Simulating reforms
  \end{enumerate}
\end{frame}

\subsection{Legislation}

\begin{frame}
  \hligne
  \frametitle{II. Model}
  \framesubtitle{\vspace{0.3cm}Legislation}\pause
  \begin{itemize}
    \item \textbf{Gathering information}
    \begin{itemize}
      \item Complex set of parameters
      \item Not easy to find for past years
    \end{itemize} \pause
    \item \textbf{The components}
    \begin{itemize}
      \item Social security contributions
      \item Income tax
      \item Local taxes
      \item Transfers
      \item Wealth and transfer taxation
      \item Corporate taxation
      \item Indirect taxation
    \end{itemize}
  \end{itemize}
\end{frame}

\begin{frame}
 \hligne
  \frametitle{II. Model}
  \framesubtitle{\vspace{0.3cm}Example of tax parameters}\pause
\end{frame}

\begin{frame}
  \hligne
  \frametitle{II. Model}
  \framesubtitle{\vspace{0.3cm}Legislation}\pause
  \begin{itemize}
    \item \textbf{Model}
    \begin{itemize}
    \item Set-up tax/transfer functions depending on input variables $x$
    $$ T = f(x)$$
    \item $f(x)$ is complex (particularly in France)
    \item e.g. 35 different SSC
    \end{itemize} \pause
    \item \textbf{Simplifications}
    \begin{itemize}
    \item Simplify the actual tax code to allow simulation
    \item e.g. SSC depend on:
    \begin{itemize}
      \item postcode of employer (transport tax)
      \item whether in Alsace-Lorraine or not
      \item prevalence of work accident by occupation
      \item size of firm
      \item share of earnings from bonus (in the public sector)
    \end{itemize}
    \end{itemize}
  \end{itemize}
\end{frame}

\begin{frame}
  \hligne
  \frametitle{II. Model}
  \framesubtitle{\vspace{0.3cm}Legislation}\pause
  \begin{table}[ht!]
\footnotesize
\caption{Different measures of earnings \label{micro}}
\vspace{0.2cm}
\begin{tabular}{|l|c|c|c|c|}
\hline
Included & Cost of & Gross & Taxable & Net \\
         & labour &        & income &  \\
\hline
Payroll tax (TS) & $\checkmark$ &\multicolumn{3}{c|}{\ } \\
\cline{1-2}
Employer SSC & $\checkmark$ &\multicolumn{3}{c|}{\ } \\
\cline{1-3}
Employee SSC & $\checkmark$ & $\checkmark$ &\multicolumn{2}{c|}{\ } \\
\cline{1-3}
CSG deductible & $\checkmark$ & $\checkmark$ &\multicolumn{2}{c|}{\ } \\
\cline{1-4}
Non ded. CSG and CRDS & $\checkmark$ & $\checkmark$ & $\checkmark$ & \\
\cline{1-5}
Net earnings after tax & $\checkmark$ & $\checkmark$ & $\checkmark$ & $\checkmark$ \\
\hline
\end{tabular}
\end{table}
\end{frame}


\subsection{Incidence}

\begin{frame}
  \hligne
  \frametitle{II. Model}
  \framesubtitle{\vspace{0.3cm}Reminder about incidence}
\vspace{1cm}
``\emph{One of the most valuable insights that economic analysis has provided in public finance is that the person who effectively \textbf{pays} a tax is not necessarily the person upon whom the tax is levied.\\ To determine the true \textbf{incidence} of
a tax or a public project is one of the most difficult, and most important, tasks of
public economics.}'' \\ \vspace{1cm} \hspace{3cm} A. Atkinson and J. Stiglitz
(1980)
\end{frame}

\begin{frame}
  \hligne
  \frametitle{II. Model}
  \framesubtitle{\vspace{0.3cm}Statutory incidence}\pause
\begin{itemize}
\item \textbf{Who pays taxes?}\\
The naive answer is that it is obvious, i.e. the person who ``pays the tax''...\pause
\begin{itemize}
  \item e.g. employees pay employees' SSC
  \item e.g. employers pay employers' SSC
  \item e.g. corporations pay corporation tax
\end{itemize} \pause
\item \textbf{Statutory/formal incidence}\\
It is the legal liability of a tax (what the law says). \pause
\item \textbf{Legal liability/remittance responsibility}
\begin{itemize}
  \item e.g. employees' SSC are paid to the tax authorities by employers
  \item e.g. income tax can also be withheld by employers
\end{itemize}
\end{itemize}
\end{frame}

\begin{frame}
  \hligne
  \frametitle{II. Model}
  \framesubtitle{\vspace{0.3cm}Economic incidence}\pause
\begin{itemize}
\item \textbf{Economic/effective incidence}
    \begin{itemize}
    \item It describes who actually bears the tax burden, i.e. who is worse off as a result of the tax. \pause
    \end{itemize}
\item \textbf{Who pays taxes in reality?}
  \begin{itemize}
  \item Taxes change prices of goods and rewards to factors
  \item e.g. an increase in employers' SSC \pause
    \begin{itemize}
    \item If net wages decrease \\ $\Rightarrow$ employees pay the tax \pause
    \item If net wages stay the same and profits go down \\$\Rightarrow$ employers pay the tax \pause
    \item If net wages stay the same and prices go up \\$\Rightarrow$ consumers pay the tax
    \end{itemize}
  \end{itemize}
\end{itemize}
\end{frame}

\begin{frame}
  \hligne
  \frametitle{II. Model}
  \framesubtitle{\vspace{0.3cm}Economic incidence}\pause
\begin{itemize}
\item \textbf{People pay taxes}
    \begin{itemize}
    \item Corporations do not pay taxes! \pause
    \item Corporation tax can affect three groups of \emph{individuals}:
    \begin{enumerate}
      \item Those who own the company (through lower profits)
      \item Those who work for the company (through lower wage)
      \item Those who consume the products of the company (through higher prices)
    \end{enumerate}
    \end{itemize}
\end{itemize}
\end{frame}

\begin{frame}
  \hligne
  \frametitle{II. Model}
  \framesubtitle{\vspace{0.3cm}Incidence assumptions in TAXIPP}\pause
  \begin{itemize}
    \item \textbf{Employer social security contributions (SSC)}
    \begin{itemize}
      \item Debate on whether incident on consumers (higher prices) or employees (lower wages)
      \item Obvious long vs short term incidence
      \item TAXIPP: incidence on employees
      \item It has consequences for analysing policy of reduction of SSC on low earners.
    \end{itemize} \pause
    \item \textbf{Indirect taxes}
    \begin{itemize}
    \item Usual analysis lead to most indirect taxes are paid by consumers
    \item Detailed analysis (Carbonnier 2007, 2009) suggest part is paid by factors
    \item TAXIPP: 70\% on prices, 30\% on factors (labour and capital)
    \end{itemize}
  \end{itemize}
\end{frame}

\begin{frame}
  \hligne
  \frametitle{II. Model}
  \framesubtitle{\vspace{0.3cm}Incidence assumptions in TAXIPP}
  \begin{itemize}
    \item \textbf{Corporate income tax}
    \begin{itemize}
      \item Standard view: CT paid by shareholders
      \item Modern finance view: likely to be paid by capital owners at large
      \item Some studies suggest that CT is paid mostly by consumers
      \item TAXIPP: incidence on capital owners except regulated interest accounts
      \item Huge implications in terms of redistribution analysis!
    \end{itemize} \pause
    \item \textbf{Undistributed corporate profits}
    \begin{itemize}
      \item Undistributed profits are part of national income
      \item CT is imposed on them
      \item Problem: who receive this income ?
      \item Shareholders are the likely recipients
    \end{itemize}
  \end{itemize}
\end{frame}

\begin{frame}
  \hligne
  \frametitle{II. Model}
  \framesubtitle{\vspace{0.3cm}Incidence assumptions in TAXIPP}
  \begin{itemize}
    \item \textbf{Payroll tax ``taxe sur les salaires'' (TS)}
    \begin{itemize}
      \item TS is a progressive payroll tax on earnings for firms in non VAT sectors (finance, insurance, education, etc.)
      \item Incidence is not obvious; no empirical studies
      \item Two options: incidence on consumers or on employees
      \item TAXIPP: incidence on employees
      \item But likely to be wrong
    \end{itemize}
  \end{itemize}
\end{frame}


\subsection{Reforms}

\begin{frame}
  \hligne
  \frametitle{II. Model}
  \framesubtitle{\vspace{0.3cm}Simulating reforms}\pause
  \begin{itemize}
    \item \textbf{Building a baseline}
    \begin{itemize}
      \item Assumption about growth rates
      \item Large implications in terms of tax revenues
    \end{itemize}\pause
    \item \textbf{No behavioural case}
    \begin{itemize}
      \item Apply the change in tax system
      \item Make comparative statistics
    \end{itemize}\pause
    \item \textbf{With behavioural response}
    \begin{itemize}
      \item Imbed an elasticity of the tax base to a change in tax rate
      \item Currently only done ad hoc for labour supply
    \end{itemize}
  \end{itemize}
\end{frame}

\begin{frame}
  \hligne
  \frametitle{II. Model}
  \framesubtitle{\vspace{0.3cm}Issues}\pause
  \begin{itemize}
    \item \textbf{Interactions between tax bases}
    \begin{itemize}
      \item Tax bases of one tax depends on other changes
      \item e.g. increase in SSC $\Rightarrow$ lower taxable income \\
      $\Rightarrow$ lower income tax
    \end{itemize} \pause
    \item \textbf{Inconsistency}
    \begin{itemize}
      \item No behavioural response is inconsistent
      \item e.g. increase in income tax $\Rightarrow$ lower consumption or/and lower savings \\
      $\Rightarrow$ lower VAT or/and lower capital taxation
    \end{itemize}
  \end{itemize}
\end{frame}

\section{III. Measuring redistribution}

\begin{frame}
  \hligne
  \frametitle{III. Measuring redistribution}
  \framesubtitle{\vspace{0.3cm}}
  \begin{enumerate}
    \item Measuring contributive capacity
    \item Household vs individual
    \item Life cycle issues
    \item Representation issues
  \end{enumerate}
\end{frame}

\subsection{Contributive capacity}

\begin{frame}
  \hligne
  \frametitle{III. Measuring redistribution}
  \framesubtitle{\vspace{0.3cm}Measuring contributive capacity}\pause
  \begin{itemize}
    \item \textbf{Types of income}
    \begin{itemize}
      \item Net incomes are not a good measure
      \item Economic income: income before all taxes
      \item Need to add all taxes to net income including imputed indirect taxes
    \end{itemize}\pause
        \item \textbf{Primary vs secondary income}
    \begin{itemize}
      \item Primary income: income before transfer and taxes
      \item Secondary income: income before all taxes but including transfer income (pensions, unemployment) but net of SSC funding these transfers
      \item Primary income $\simeq$ national income
    \end{itemize}
  \end{itemize}
\end{frame}

\begin{frame}
  \hligne
  \frametitle{III. Measuring redistribution}
  \framesubtitle{\vspace{0.3cm}Measuring contributive capacity}\pause
  \begin{itemize}
    \item \textbf{Income vs consumption}
    \begin{itemize}
      \item Income might not be a good measure of permanent income
      \item Temporary variations in income are frequent\\
      e.g. unemployment: primary income drops to zero
      \item Consumption might be a better measure of permanent income
      \item Except that consumption does not capture systematic difference in savings over the life-cycle
    \end{itemize}\pause
    \item \textbf{Income vs wealth}
    \begin{itemize}
    \item Income and wealth are not completely correlated
    \end{itemize}\pause
    \item \textbf{Philosophical backgrounds}
    \begin{itemize}
    \item Social welfare functions depend on utilities
    \item Is utility derived from income, consumption, wealth...
    \end{itemize}
  \end{itemize}
\end{frame}

\subsection{Household vs individual}

\begin{frame}
  \hligne
  \frametitle{III. Measuring redistribution}
  \framesubtitle{\vspace{0.3cm}Household vs individual}\pause
  \begin{itemize}
    \item \textbf{What unit of reference?}
    \begin{itemize}
      \item Individuals: income
      \item Household: income is totally pooled among household members
      \item Partly a philosophical choice
    \end{itemize}\pause
        \item \textbf{How to account for household size?}
    \begin{itemize}
      \item Assessment of needs represented by each member
      \item OECD equivalence scale
    \end{itemize}
  \end{itemize}
\end{frame}

\subsection{Life cycle issues}

\begin{frame}
  \hligne
  \frametitle{III. Measuring redistribution}
  \framesubtitle{\vspace{0.3cm}Life cycle issues}\pause
  \begin{itemize}
    \item \textbf{Cross-section is misleading}
    \begin{itemize}
      \item Data source essentially cross-section
      \item Redistribution analysis is misleading
    \end{itemize}\pause
    \item \textbf{Life-cycle issues}
    \begin{itemize}
    \item Redistribution through contributory pensions/unemployment
    \item Age earnings profile explains part of the earnings inequality
    \end{itemize}\pause
    \item \textbf{Income variability}
    \begin{itemize}
    \item Income shocks from one year to the other
    \item Bottom of the distribution: the ``poor'' are the unemployed or people at the minimum wage?
    \end{itemize}
  \end{itemize}
\end{frame}

\subsection{Representation issues}

\begin{frame}
  \hligne
  \frametitle{III. Measuring redistribution}
  \framesubtitle{\vspace{0.3cm}Representation issues}\pause
  \begin{itemize}
    \item \textbf{Choice of redistributive capacity}
    \begin{itemize}
      \item Current income
      \item Measure of permanent income
      \item Consumption
    \end{itemize}\pause
    \item \textbf{Choice of unit of reference}
    \begin{itemize}
    \item Individuals, households, consumption unit
    \end{itemize}\pause
    \item \textbf{Scale of the distribution}
    \begin{itemize}
    \item Quintile, decile, percentile
    \item Absolute values
    \end{itemize}
  \end{itemize}
\end{frame}


\begin{frame}
  \hligne
  \frametitle{III. Measuring redistribution}
  \framesubtitle{\vspace{0.3cm}Representation issues}
\end{frame}

\begin{frame}
  \hligne
  \frametitle{III. Measuring redistribution}
  \framesubtitle{\vspace{0.3cm}Representation issues}
\end{frame}

\begin{frame}
  \hligne
  \frametitle{III. Measuring redistribution}
  \framesubtitle{\vspace{0.3cm}Representation issues}
\frametitle{Taux de pr�l�vements obligatoires}

\end{frame}

\section{IV. Future developments}

\begin{frame}
  \hligne
  \frametitle{IV. Future developments}
  \begin{itemize}
      \item \textbf{Part of larger set of models in development}
      \begin{itemize}
        \item PENSIPP: dynamic microsimulation model of the French pension system
        \item TAXIPP-LIFE: over the life-cycle
      \end{itemize}
    \item \textbf{Development of data sources}
    \begin{itemize}
      \item Setting-up data from different years
      \item Developing matching procedures for other sources
    \end{itemize} \pause
    \item \textbf{Simulating past tax systems}
    \begin{itemize}
      \item Going back to the 1970s...
    \end{itemize} \pause
     \item \textbf{Developing the behavioural module}
    \begin{itemize}
      \item More systematic module of behavioural responses
      \item Including substitution responses and real responses
    \end{itemize} \pause
      \item \textbf{Developing other modules}
    \begin{itemize}
      \item Firm side
      \item Public spending
      \item TAXIPP-LIFE
    \end{itemize} \pause
     \item Translating into R or Python?
  \end{itemize}
\end{frame}

\frame{\maketitle}

\end{document}
