\documentclass[11pt,a4paper]{article}
\usepackage[latin1]{inputenc}
\inputencoding{latin1}
\usepackage[T1]{fontenc}
\usepackage{amsmath}
\usepackage{amsfonts}
\usepackage{amssymb}
\usepackage{graphicx}
\usepackage{verbatim}
\usepackage{multicol}
\usepackage{setspace}
\usepackage{makeidx}
\usepackage{epigraph}
\usepackage{hyperref}
\onehalfspacing
\usepackage[french]{babel}
\usepackage{fullpage}
 %\voffset=-2.5cm
 %\hoffset=-2cm
% \textwidth=17cmrr
 %\textheight=24.5cm
 \newcommand{\til}{TaxIPP-Life}

\author{B�atrice Boutchenik et Alexis Eidelman\\ IPP}
\title{Micro-simulation sur le cycle de vie :\\ le mod�le TaxIPP-Life}


\begin{document}
\maketitle

\begin{center}
Document extr�mement pr�liminaire. Ne pas diffuser, ne pas citer.  \newline \newline
Ce document pr�sente le mod�le \til. La construction de ce mod�le a d�but� il y a moins d'un an au moment de la r�daction de ce rapport. Le mod�le est encore dans une phase de d�veloppement. Les r�sultats du mod�le correspondent pour l'instant plus � des exercices de test qu'� des statistiques. 
\end{center}


\newpage
\tableofcontents
\newpage

\section*{Introduction}
\subsection*{Motivations}
\addcontentsline{toc}{subsection}{Motivations}
L'objet du projet \til, d�but� en septembre 2012, est la r�alisation d'�tudes sur la population fran�aise portant sur l'ensemble du cycle de vie. Le regard que l'on porte sur les in�galit�s peut se voir modifi� dans cette optique.
En effet, Lorsqu'on s'int�resse � des questions empiriques touchant aux in�galit�s et � la redistribution, dont les r�ponses d�pendent en grande partie de la distribution effective des revenus, le fait de se concentrer sur une ann�e donn�e seulement a de fortes chances de mener � une vision erron�e des ph�nom�nes �tudi�s.
Premi�rement, les revenus des individus peuvent en effet �tre amen�s � varier de fa�on consid�rable au cours de leur vie, et la mobilit� de ceux-ci dans la distribution g�n�rale des revenus aura des cons�quences en termes de mesure des in�galit�s, et donc dans la r�duction de celles-ci par des transferts dont le ciblage ne d�pend en g�n�ral que de l'ann�e en cours. 
Deuxi�mement, l'impact redistributif des diff�rents �l�ments du syst�me fiscalo-social est moins bien appr�hend� lorsque l'on fait une �tude en coupe. 
En effet, lorsque l'on �tudie une ann�e donn�e, on manque le fait qu'une partie des pr�l�vements sur les revenus d'un individu lui reviendront ou lui ont �t� avanc� � d'autres moment de sa vie. 
Pareillement, une partie des prestations sont financ�e par l'individu au cours de sa vie. 
Un �tude sur le cycle de vie pourra s�parer une partie intra-personnelle de la redistribution (au cours du temps) et une partie inter-personnelle qui correspond � une sorte de vraie redistribution des revenus de certains vers d'autres. 
L'analyse sur cycle de vie peut aussi tenir compte des transferts dit assurantiels contrairement aux �tudes en coupe qui sont contraintes � les laisser hors champs. \\ 

Les revenus per�us par les individus sont susceptibles d'�voluer au cours du temps selon deux m�canismes principaux. 
D'une part selon une perspective de cycle de vie traditionnelle dans laquelle, apr�s une p�riode de formation, les individus ont des revenus faibles en d�but de vie, puis des revenus de plus en plus importants durant leur carri�re et enfin une retraite pendant laquelle leurs revenus sont diminu�s.
D'autre part, des chocs, parfois impr�visibles, ch�mage, maladie, d'inactivit�, etc., peuvent affecter temporairement les revenus. 
Ces deux facteurs font que les individus restent rarement toute leur vie au m�me point de la distribution globale des revenus. \cite{Slemrod1992} montre par exemple dans le cas des �tats-Unis qu'entre 1980 et 1986, lorsque l'on consid�re une ann�e donn�e, plus de 20 \% des individus situ�s dans le d�cile de revenus le plus �lev� n'y �taient pas l'ann�e pr�c�dente. Ce chiffre est de 33 \% lorsqu'on consid�re le percentile le plus haut, entre 1982 et 1985. Il met par ailleurs en �vidence le fait que pour les cat�gories de revenus les plus faibles en 1983 (allant jusqu'� 25000 \$ annuels), le revenu moyen sur la p�riode 1979-1985 est syst�matiquement plus �lev� que le revenu pour la seule ann�e 1983. Ceci est d'autant plus vrai que la classe de revenu est faible : les individus dont les revenus en 1983 �taient compris entre 0 et 5000 \$ ont un revenu moyen de 3063 \$ en 1983, contre un revenu moyen de 7395 \$ sur la p�riode consid�r�e.

\cite{Bjorklund1993} examine dans le cas de la Su�de la relation entre les revenus en coupe et sur le cycle de vie, de 1951 � 1989. 
Les corr�lations entre ces deux revenus sont globalement faibles. 
Elles sont m�me presque nulles, lorsqu'on consid�re les individus � un �ge situ� entre 25 et 30 ans, alors qu'elles sont relativement fortes pour des groupes plus �g�s. 
Le revenu annuel serait donc un indicateur particuli�rement mauvais du revenu sur le cycle de vie pour les jeunes.\\

Cette variabilit� des revenus au cours de la vie des individus porte un certain nombre de cons�quences quant � la mesure des in�galit�s. 
Si les individus se situent � diff�rents points de la distribution de revenus � diff�rents moments de leur vie, il en d�coule que la vision des in�galit�s prise en coupe sera exag�r�e par rapport aux in�galit�s de revenu sur le cycle de vie. 
Consid�rons par exemple une �conomie d'individus parfaitement identiques et ayant le m�me profil de revenus au cours de leur cycle de vie (et ce dans un monde sans croissance r�elle des revenus au cours du temps), mais qui diff�rent par leur �ge : on constate l'existence d'in�galit�s pour une ann�e donn�e, alors m�me que chaque individu aura re�u � chaque �ge un revenu r�el identique � celui re�u par les autres au m�me �ge. \cite{Fitzgerald1990} ont ainsi calcul� que le coefficient de Gini calcul� pour les �tats-Unis en l'ann�e 1979 se voyait r�duit de 19,1 \% apr�s avoir contr�l� par le mouvement d'ascension des revenus selon l'�ge, � travers les cohortes. 

D'autres �tudes empiriques ont �tudi� les in�galit�s de revenus consid�r�s sur des p�riodes plus ou moins longues, en prenant donc en compte � la fois l'�volution � naturelle � des revenus sur le cycle de vie et les chocs transitoires pouvant affecter ceux-ci. \cite{Slemrod1992} rapporte les r�sultats de \cite{Shorrocks1978} et de \cite{Bemis1975} qui montrent que l'extension � deux ann�es ou plus de la p�riode sur laquelle les in�galit�s sont �tudi�es change tr�s peu la mesure de celles-ci. 
Il s'int�resse lui-m�me � la fraction de revenus d�tenue par les 1 \%, 5 \% et 10 \% mieux lotis de la population, selon qu'on consid�re une ann�e donn�e ou la p�riode allant de 1979 � 1985. 
Celle-ci passe respectivement de 8.39 � 7.64 \%, de 19.94 � 18.70 \% et de 29.96 � 28.50 \% pour les trois cat�gories d'individus consid�r�es.

Au contraire, selon \cite{Pettersson2007} qui utilisent des donn�es simul�es, le coefficient de Gini diminue de 60 \% si l'on consid�re tout le cycle de vie, par rapport � la m�me mesure pour une ann�e donn�e. 
cite{Bjorklund1993} estime que cette diminution est de l'ordre de 35 � 40 \%, en consid�rant des donn�es su�doises effectivement observ�es entre 1951 et 1989, soit une p�riode bien plus longue que celle �tudi�e par Slemrod\footnote{Il est utile de pr�ciser ici qu'une r�vision � la baisse de la mesure des in�galit�s n'implique pas de porter une attention r�duite aux bas revenus, qui m�me s'ils sont temporaires correspondent quand m�me � des situations d�favorables dans l'imm�diat, en l'absence d'acc�s parfait aux march�s financiers, et doivent �tre corrig�s}. 
Il est logique que la dur�e de la p�riode consid�r�e ait un effet important sur les conclusions auxquelles on parvient en termes d'in�galit�s et de redistribution, ce sur quoi on reviendra par la suite. 

Les �tudes portant sur les in�galit�s annuelles de revenus m�langent les variations de revenus au cours du temps pour un m�me individu et les in�galit�s entre revenus individuels totaux calcul�s sur la p�riode consid�r�e, en leur accordant le m�me poids. 
M�me si la situation � instant donn� n'est pas sans int�r�t, en particulier si l'on pense aux personnes ayant des revenus faibles temporairement dans un contexte o� l'acc�s au march� financier n'est pas parfait, ne pas tenir compte de l'�ge o� des revenus sur l'ensemble du cycle de vie est une limite pour ces �tudes. 
Les analyses portant au contraire sur une p�riode plus longue permettent au contraire de dissocier, au sein d'une mesure totale des in�galit�s, une partie tenant r�ellement aux in�galit�s entre individus (in�galit�s �inter-individuelles �) et une partie tenant � la variabilit� des revenus sur le cycle de vie (in�galit�s �intra-individuelles �). 

Toutes les �tudes montrent, avec des ampleurs diff�rentes, que la mesure des in�galit�s mesur�e en coupe est sur�valu�e par rapport � son estimation sur le cycle de vie. Si l'on croit que la consommation peut �tre liss�e au cours de la vie et que le revenu, c'est bien un revenu moyen sur plusieurs ann�es dont il faut tenir compte pour estimer le niveau de vie d'un individu.  
\\

La question du patrimoine est elle aussi primordiale dans l'�tude des in�galit�s. 
Il est en effet raisonnable de croire qu'un individu ayant un capital (�ventuellement un capital attendu � travers un h�ritage) n'aura pas la m�me attitude de consommation qu'une personne qui n'en a pas, ce que l'on traduit par des niveaux de vie diff�rent. 
Si on pense � deux �tudiants, l'un aid� par ses parents, l'autre non, celui aid� par ses parents est g�n�ralement consid�r� comme �tant plus ais� que l'autre. 
Pourtant, si pour subvenir � ses besoin le second travaille en m�me temps que ses �tudes, il sera paradoxalement consid�r� comme �tant plus riche. 
De fa�on g�n�rale, la \og richesse \fg \ d'un individu est un m�lange de ses revenus et de son patrimoine, les �tudes en coupe sont encore limit�e parce que le patrimoine, parce qu'il est parfois peu liquide ou pr�cis�ment parce qu'il ne peut s'expliquer que par des consid�rations sur cycle de vie, est exclu de l'analyse. 
Le revenu potentiel qu'il repr�sente n'est pourtant pas n�gligeable et la diff�rence entre un propri�taire et un locataire se situe � la fois dans le loyer (et ce point devrait toujours �tre r�gl� dans les �tudes d'in�galit�s) et � la fois dans le revenu actualis� que pourrait repr�senter la vente du logement. \\
 
Le premier point pour justifier une approche sur cycle de vie est donc que l'on peut mieux appr�hender la situation des individus. Un deuxi�me est que si l'on veut �tudier la progressivit� des transferts ou du syst�me socio-fiscal dans son ensemble, une �tude en coupe souffre de limitation importante. \\

Le constat port� sur le revenu en coupe de individu qui n'est pas suffisant pour appr�hender leur richesse tient aussi pour la redistribution. 
Une analyse en coupe, sans �tre d�nu�e d'int�r�t, ne permet pas de r�aliser qu'une partie des transferts correspondent moins � de la redistribution qu'� un lissage forc� des revenus, ou a une consommation forc�e (d'�ducation par exemple mais aussi d'assurance) dont l'individu b�n�ficie personnellement.
Si on reprend l'exemple th�orique d'une population d'individus parfaitement identiques et ayant le m�me profil de revenus au cours de leur cycle de vie et qu'un imp�t est pr�lev� pour financer un certain transfert, par exemple un imp�t sur le revenu pour financer des retraite ou de l'�ducation, une ann�e donn�e, il y aura un transfert des actifs vers les jeunes et les retrait�s mais dans cette population, toutes les carri�res sont les m�mes si bien que l'ensemble des pr�l�vements sur les revenus d'un individu lui reviendront sous forme de retraite ou de d�pense d'�ducation.  \\


Empiriquement, \cite{Bengtsson2012} montrent ainsi que la progressivit� du syst�me fiscal su�dois sur le cycle de vie est plus faible que sa progressivit� pour une ann�e donn�e, ce pour presque toutes les ann�es de la p�riode 1968-2009. 
Ainsi le syst�me serait presque proportionnel et aurait un effet redistributif de seulement quelques points de pourcentage, r�duisant le coefficient de Gini de 10 \%. 
\cite{Bjorklund1993} met quant � lui en �vidence une r�duction de celui-ci de 20 \% sur la p�riode 1978-1990, pour la Su�de �galement, ce qui montre l'importance de l'�tendue de la p�riode �tudi�e. 
Pour les Etats-Unis, \cite{Slemrod1992} trouve un effet redistributif compris entre 4 et 6 \% lorsqu'on consid�re le cycle de vie.
A titre d'exemple, le premier d�cile est tax� en moyenne � hauteur de 4.6 \% lorsqu'on consid�re une ann�e donn�e, contre 6.1 \% sur le cycle de vie, le deuxi�me � 5.5 et 7.4 \% respectivement pour une ann�e et sur le cycle de vie. Les �carts sont bien moins grands dans le haut de la distribution. 
De m�me,cite{Bengtsson2012} montrent que ce sont surtout les quatre premiers quintiles qui sont affect�s par la perspective de cycle de vie, alors que celle-ci ne change pas de beaucoup le taux de taxation effectif pour le quintile de revenus le plus �lev�. \\

Ces travaux am�nent donc bien � nuancer la redistribution effectivement op�r�e entre individus par les syst�mes fiscalo-sociaux existants. Celle-ci ne serait donc pas uniquement � inter-personnelle �, mais �galement � intra-personnelle �.
Ainsi \cite{Pettersson2007} estiment qu'entre 18 et 32 \% de la redistribution effectu�e par le syst�me de redistribution su�dois (taxes et prestations mon�taires et non-mon�taires) est effectivement de la redistribution entre individus, le reste correspondant � de la redistribution intra-personnelle qui est donc autofinanc�e : environ 7 � 8 couronnes su�doises sur 10 re�ues sont pay�es, en moyenne, par les individus eux-m�mes � un autre moment. 
Dans le cas de la Su�de, l'effet principal du syst�me de redistribution serait donc un transfert de ressources au sein du cycle de vie. Les auteurs rapportent les r�sultats similaires de \cite{Hussenius1994} qui pour ce m�me pays estiment que le taux de redistribution inter-personnelle se situe entre 24 et 32 \%.
�tudiant le cas de l'Australie, \cite{Falkingham1996} situent ce chiffre entre 48 et 63 \%, et entre 29 et 38 \% pour le Royaume-Uni. Il est par ailleurs estim� � 45 \% pour l'Irlande et � 24 \% pour l'Italie \cite{O'Donoghue2001}.\\

La part de redistribution intra-personnelle des syst�mes fiscalo-sociaux des diff�rents pays pour lesquels elle a �t� mesur�e appara�t donc substantielle. 
Celle-ci refl�te une logique d'assurance contre les p�riodes du cycle de vie o� les individus voient leurs revenus baisser ou certains types de d�penses augmenter, et ce de fa�on plus ou moins pr�visible. 
\cite{Varian1980} a ainsi soulign� d�s 1980 que la taxation peut �tre consid�r�e comme une assurance sociale, ayant pour r�le d'�galiser le revenu et la consommation sur le cycle de vie. 
L'ensemble du syst�me redistributif participe donc potentiellement de la fonction assurantielle de l'Etat, dans un monde o� les march�s financiers ne sont pas toujours complets. 
On aura pu noter dans le paragraphe pr�c�dent que les taux de redistribution inter-personnelle (rapport�e � la redistribution totale) sont relativement plus �lev�s dans les pays anglo-saxons qu'en Su�de, par exemple. 
Ceci est � rapprocher du constat de \cite{Lindert2004} selon lequel les pays ayant un ratio �lev� de d�penses publiques rapport�es au PIB ont �galement un syst�me fiscal plus proportionnel, comparativement aux pays anglo-saxons. Dans ces derniers, la fonction d'assurance est plut�t assur�e par les march�s, ce qui laisse � l'Etat celle d' � assistance �. 
Ceci est �galement coh�rent avec la taxonomie canonique de l'Etat-Providence (\cite{Esping-Andersen1990}), dans laquelle les Etats scandinaves sont caract�ris�s par un haut niveau d'universalit� et une d�pendance limit�e aux march�s et � la famille.\\
	
La distinction entre redistribution inter-personnelle et redistribution intra-personnelle (ou assurantielle) rev�t donc une grande importance dans le cadre du d�bat public. D'une part lorsqu'il s'agit de comparer par exemple les taux de pr�l�vements obligatoires entre diff�rents pays : il est n�cessaire de garder � l'esprit que les �carts observ�s peuvent s'expliquer pour une large partie par le caract�re public ou priv� des fonctions d'assurance. D'autre part, cette distinction est cruciale dans la r�flexion men�e sur l'�quit� des syst�mes redistributifs : non seulement la redistribution effective entre individus doit �tre corrig�e de la surestimation qui intervient lorsqu'on se restreint � une seule ann�e, mais il est �galement important de prendre en compte les cons�quences en termes de bien-�tre de la redistribution intra-personnelle, cons�quences qui peuvent �tre in�galement r�parties au sein de la population. 
C'est ce que montrent \cite{Hoynes2011} dans le cas des Etats-Unis : la valeur de la composante assurantielle de la redistribution est positive pour l'ensemble de la population, mais elle est d'autant plus grande que le revenu est �lev�.\\

Tous les �l�ments des syst�mes de redistribution ne contribuent pas de fa�on �gale aux composantes de redistribution intra- et inter-personnelle de ceux-ci. 
\cite{Bjoerklund1997} montrent ainsi que l'imp�t sur le revenu su�dois agit principalement sur les in�galit�s entre individus (consid�r�s sur le cycle de vie), alors que l'allocation familiale universelle a un effet � la fois sur celles-ci et sur les in�galit�s au sein du cycle de vie pour chaque individu. Ceci est tout � fait logique puisque les auteurs prennent en compte la consommation des enfants � travers une �chelle d'�quivalence, et que le but des allocations familiales est d'apporter un soutien �conomique dans les p�riodes de plus lourdes responsabilit�s familiales. 
L'allocation logement su�doise, accord�e sous condition de ressources contrairement aux allocations familiales, poss�de elle aussi un effet � la fois sur les in�galit�s inter- et intra-personnelles. Les auteurs montrent par ailleurs que la volatilit� des revenus est la plus forte pour les individus ayant les plus bas revenus totaux sur le cycle de vie, et que c'est donc sur leurs revenus que le syst�me redistributif exerce le plus fort lissage. 
De m�me, la redistribution intra-personnelle que mettent en �vidence \cite{Bengtsson2012} refl�te d'apr�s eux la nature transitoire de la perception de bas revenus plut�t que la nature transitoire des hauts revenus. La perspective de cycle de vie serait donc particuli�rement importante lorsqu'on consid�re les �l�ments redistributifs touchant les individus � bas revenus : c'est pour ces composantes qu'on pourrait observer l'effet assurantiel le plus fort relativement � l'effet r�ellement redistributif.\\

\subsection*{Les difficult�s d'une �tude sur le cycle de vie}
\addcontentsline{toc}{subsection}{Les difficult�s d'une �tude sur le cycle de vie}

L'�tude sur le cycle de vie exige, et c'est une lapalissade, des donn�es sur l'ensemble de ce cycle de vie. La base de donn�es souhaitable serait un panel sur l'ensemble du cycle de vie contenant toutes les informations utiles sur le patrimoine, les revenus, les statuts conjugaux, la consommation, etc. Si des donn�es proches de cet id�al existe dans les pays scandinaves, ce n'est pas le cas en France. En interrogeant un individu une ann�e donn�e, pour connaitre son cycle de vie s'expose � deux probl�me. Le plus �vident est que lorsque l'on interroge un individu, il ne connait pas son avenir. Le mod�le \til \ contient un module r�alisant une projection � un niveau individuel du futur. Ceci sera d�velopp� dans la deuxi�me partie de ce document. La premi�re s'attardera sur un point moins �vident qui est que le pass� des individus n'est pas non plus connu parfaitement. En dehors des effets de m�moire qui font qu'une partie du pass� est oubli�e, aucune enqu�te ne demande � un individu de d�tailler l'ensemble de ses revenus pass�s. Des donn�es administratives semblent �tre la seule solution pour connaitre par exemple la trajectoire professionnelle des individus. Mais les donn�es administratives ne sont pas non plus une panac�e, elles ne donnent en g�n�ral pas d'information sur la vie matrimoniale des personnes, or, savoir qui vit avec qui est d�terminant pour �tudier la consommation et les niveaux de vie. Enfin, une troisi�me voie serait l'utilisation d'un panel qui suivrait les individus depuis leur naissance ou du moins pendant un certain temps, toutefois une telle base de donn�es, avec des informations assez riches pour pouvoir appliquer la l�gislation socio-fiscal n'existe pas. Nous avons donc proc�d� � un matching statistique de donn�es d'enqu�te et de donn�es administratives. La premi�re partie de document d�crira ce matching avec entre les donn�es de l'enqu�te patrimoine et les donn�es de l'�chantillon EIR-EIC. 

Enfin, la troisi�me partie de ce document pr�sentera des r�sultats en ce centrant sur les prestations. Si l'objectif du projet \til \ est, � terme, de simuler l'ensemble du syst�me de redistribution fran�ais sur le cycle de vie. Il para�t toutefois int�ressant de se concentrer dans un premier temps sur les prestations sociales. Celles-ci op�rent en effet une grande partie de la redistribution observ�e en coupe en France. D'apr�s \cite{Duval2011}, elles contribuent pour deux tiers � la r�duction des in�galit�s en France en 2011, contre un tiers pour les pr�l�vements. De plus, elles sont l'instrument le plus susceptible d'avoir un impact sur les individus � bas revenus \footnote{Le rapport \cite{Bourguignon1998} montre ainsi que les instruments dominants de la redistribution sont les transferts sous condition de ressources dans le bas de la distribution, et la progressivit� de l'imp�t sur le revenu dans le haut de la distribution.}, dont on a vu qu'ils connaissent la volatilit� de revenus sur le cycle de vie la plus �lev�e. C'est donc la redistributivit� des prestations sociales qui risque d'�tre la plus affect�e par la perspective de cycle de vie. Enfin, dans le cadre du d�bat actuel portant sur l'�quilibre de la branche famille de la S�curit� sociale et apr�s la publication du rapport \cite{Fragonard2013}, un �clairage dans une perspective de cycle de vie sur certaines propositions de r�formes, notamment en ce qui concerne les allocations familiales, para�t particuli�rement int�ressant.\\



\section{Construction de la base de donn�es}



\subsection{Appariement statistique des donn�es EIC-DADS avec l'Enqu�te Patrimoine}

Un des int�r�ts majeurs de TAXIPP Life r�side dans l'utilisation de trajectoires salariales effectivement observ�es, plut�t que simul�es. Les donn�es administratives des EIC et DADS ont �t� utilis�es � cette fin. Celles-ci ont �t� coupl�es � l'enqu�te Patrimoine afin de disposer d'un certain nombre d'informations, notamment au niveau du m�nage plut�t que de l'individu. Un appariement statistique a donc �t� effectu� entre les deux sources de donn�es : appariement statistique exact sur un certain nombre de variables consid�r�es au moment de l'enqu�te (sexe, �ge, PCS � 1 chiffre, tranche de revenus salariaux et de remplacement), puis appariement avec le plus proche voisin, en particulier en termes de trajectoire vis-�-vis de l'emploi.\\

Il a �t� dans un premier temps n�cessaire de manipuler les fichiers EIC-DADS (et UNEDIC) d'une part et l'enqu�te Patrimoine d'autre part afin de les rendre plus exactement comparables quant aux variables utilis�es pour l'appariement statistique : il faut en effet que ces variables soient d�finies de la fa�on la plus proche possible dans une base et dans l'autre. En particulier, il a fallu utiliser l'information des fichiers EIC-DADS-UNEDIC afin de retracer les trajectoires vis-�-vis de l'emploi, selon les m�mes modalit�s que celles renseign�es dans le calendrier r�trospectif de l'Enqu�te Patrimoine, calendrier renseignant les changements entre des situations ayant dur� au moins un an.

\subsubsection{Mise en forme des fichiers EIC-DADS-UNEDIC}

On conserve uniquement les individus n�s � partir de 1942, les donn�es concernant les g�n�rations 1934 et 1938 n'�tant pas compl�tes.

\paragraph{Reconstitution des trajectoires vis-�-vis de l'emploi}
Les observations du fichier DADS correspondent au croisement individu x ann�e x entreprise. A partir de ce fichier et pour les ann�es 1976 � 2001, on d�termine le statut de l'individu au 1er janvier de chaque ann�e, en termes d'emploi � temps plein ou temps partiel. Les donn�es du fichier EIC caisse x individu x ann�e permettent par ailleurs de croiser cette information avec le fait que le salari� �tait dans le public, le priv� ou encore �tait ind�pendant pour l'ann�e donn�e. Ces sources permettent de retracer les modalit�s suivantes de la variable de statut vis-�-vis de l'emploi de l'Enqu�te Patrimoine : salari� du public � temps complet, salari� du public � temps partiel, salari� du priv� � temps complet, salari� du priv� � temps partiel et � son compte .
Lorsqu'il y a cumul de plusieurs emplois au premier janvier de l'ann�e, que ceux-ci soient � temps plein ou temps partiel, on consid�re que l'individu travaille � temps plein. Enfin, l'appartenance au public  ou � la cat�gorie des ind�pendants �tant obtenue dans l'EIC gr�ce � l'appartenance � diff�rentes caisses - appartenance dont seule l'ann�e est connue et non les dates exactes au cours de cette ann�e -, on a 2116 individus qui pour au moins une ann�e sont affili�s � la fois � une caisse du public et � une caisse d'ind�pendants. On attribue al�atoirement un des deux statuts � ces individus pour l'ann�e donn�e.

Le fichier UNEDIC permet quant � lui de retracer pour les ann�es 1984 � 2001 et gr�ce au type d'allocation per�ue le fait qu'au 1er janvier de chaque ann�e l'individu soit en situation de ch�mage indemnis� ou non (ch�mage indemnis� seulement jusqu'en 1992), de formation et de pr�retraite. S'il y a cumul de plusieurs allocations dont une de pr�retraite ou de formation au 1er janvier de l'ann�e consid�r�e, on classe l'individu comme �tant plut�t en pr�retraite ou formation. Lorsqu'on a � la fois un statut provenant du fichier DADS et un statut provenant du fichier UNEDIC pour le m�me individu et la m�me ann�e, on conserve le statut DADS.
L'information provenant du fichier EIC caisse x individu x ann�e , compl�t�e par les dates de sortie des fichiers EIC-DADS-UNEDIC, nous permet de retracer les d�parts � la retraite.
Enfin, le calendrier r�trospectif de l'enqu�te Patrimoine comprend une modalit� "succession de courtes p�riodes  (inf�rieures � un an) d'emploi et de ch�mage", qu'il serait dommage d'assimiler al�atoirement � de l'emploi ou du ch�mage, celle-ci nous renseignant sur un type de situation particuli�re vis-�-vis de l'emploi. De telles p�riodes ont ainsi �t� rep�r�es dans le fichier DADS.

\paragraph{Revenus salariaux et de remplacement pour l'ann�e 2001}
On additionne les salaires nets renseign�s dans les DADS et les salaires nets des fonctionnaires de l'Etat donn�s dans la base EIC caisse x individu x ann�e. On prend enfin en compte les revenus de remplacement, que l'on divise par 0,7 afin de comparer des salaires de r�f�rence et non des revenus effectifs, qui nous int�resseraient moins ici : il serait par exemple peu souhaitable d'apparier un individu gagnant 2000 euros mensuels avec un autre gagnant g�n�ralement 3000 euros environ, mais qui l'ann�e consid�r�e se trouve �tre en situation de ch�mage et donc le revenu de remplacement est donc plus proche de 2000 euros.\\





\subsubsection{Mise en forme de l'Enqu�te Patrimoine}

	L'appariement se fera naturellement lorsque l'on disposera du fichier EIC-DADS 2010, mais on matche pour l'instant l'Enqu�te Patrimoine 2009-2010 et l'EIC 2001 en alignant la premi�re sur l'EIC, et en consid�rant donc la situation d'un individu pour l'ann�e 2010 dans l'Enqu�te Patrimoine comme correspondant � celle pour l'ann�e 2001 de l'EIC. Ainsi, on utilise pour l'appariement selon les trajectoires des dates retard�es de 9 ann�es pour l'Enqu�te Patrimoine, ce qui revient � comparer des situations vis-�-vis de l'emploi pour un �ge donn�.
On d�flate par ailleurs les revenus salariaux et de remplacement de l'Enqu�te Patrimoine 2010 par la croissance nominale observ�e des salaires afin qu'ils soient comparables � ceux provenant des fichiers EIC-DADS-UNEDIC en 2001. De m�me que pour les donn�es issues du fichier EIC, on multiplie les revenus de remplacement par un facteur de 0.7 avant de les additionner aux revenus salariaux.

Les donn�es EIC �tant disponibles pour une g�n�ration sur quatre, on regroupe �galement les individus de l'Enqu�te Patrimoine en cat�gories correspondant � quatre ann�es de naissance : les individus de l'Enqu�te Patrimoine n�s entre 1941 et 1944 seront par exemple appari�s avec la g�n�ration n�e en 1942 du fichier EIC, ceux n�s entre 1945 et 1948 avec la g�n�ration 1946, et ainsi de suite. L'�chantillon est donc compos� d'individus n�s entre 1941 et 1972.\\

On conserve pour l'appariement statistique uniquement les individus qui au moment de l'enqu�te ne sont pas agriculteurs ou ind�pendants, et qui dans leur trajectoire r�trospective de l'Enqu�te Patrimoine ont d�clar� au moins une p�riode comme salari� (� temps plein ou temps partiel, dans le public ou le priv�), comme au ch�mage ou en alternance de courtes p�riodes d'emploi et de ch�mage, entre 1984 et 2001. Pour ceux qui n'ont pas connu de tels �pisodes sur cette p�riode, on matche s�par�ment ceux qui en ont connu au moins une sur la p�riode 1976 � 1983. Les autres ne sont donc pas appari�s et se verront attribuer des revenus salariaux nuls entre 1976 et 2001, ce qui est coh�rent leur d�claration dans l'enqu�te Patrimoine.
On dissocie en effet 1976-1983 et 1984-2001 car la variable de statut vis-�-vis de l'emploi ne peut �tre d�finie de la m�me mani�re sur ces deux p�riodes, les donn�es UNEDIC n'�tant pas renseign�es avant 1984.

Enfin, on n'apparie pas pour l'instant les individus � la retraite \footnote{On ne dispose pas de suffisamment d'individus � la retraite dans l'�chantillon EIC 2001 (puisqu'on ne prend en compte que les g�n�rations n�es � partir de 1942).}.



\subsubsection{Appariement des deux bases}

L'appariement est effectu� de fa�on exacte sur les variables de sexe, d'�ge par tranches de quatre ans, de PCS � un chiffre et de revenus salariaux et de remplacement par tranches au nombre de 12, entre les donn�es de l'ann�e 2001 dans le fichier EIC-DADS et celles de l'ann�e 2010 pour l'Enqu�te Patrimoine (ces derni�res �tant r�-�valu�es pour les revenus salariaux et de remplacement). Au sein d'une cellule d'individus poss�dant les m�mes caract�ristiques, on recherche alors un plus proche voisin quant � la trajectoire en termes de statut entre 1984 et 2001, sauf pour les individus n'ayant pas eu de p�riode salari�e ou de ch�mage entre 1984 et 2001, et aux revenus salariaux et de remplacement.

La distance entre trajectoires est calcul�e en sp�cifiant des co�ts de suppression d'un statut - il est ainsi co�teux de passer d'une trajectoire � la trajectoire identique, avec simplement une p�riode en moins ou en plus - et de substitution d'un statut � un autre. Le co�t de suppression est le m�me quel que soit le statut pendant la p�riode supprim�e, mais les co�ts de substitution varient selon le statut que l'on substitue � un autre statut donn� : on doit ainsi d�finir une matrice de co�ts (cf. infra). On n'autorise pas les statuts manquants au cours ou � la fin d'une trajectoire : une p�riode manquante est cat�goris�e comme "Inactif et autres". On autorise par contre les statuts manquants au d�but des trajectoires, c'est-�-dire avant l'entr�e dans une des bases de donn�es, si celle-ci a lieu apr�s 1976. Une trajectoire consid�r�e est ainsi plus courte si elle comporte des statuts manquants au d�but, mais cela n'est pas vrai si elle comporte des statuts manquants au milieu ou � la fin.


\paragraph{D�finition de la matrice de co�ts pour la distance entre trajectoires}
 La distance entre deux statuts est d�finie comme �tant la diff�rence en valeur absolue entre les revenus (d'activit� et de remplacement) m�dians parmi les sous-groupes se trouvant dans chacun des deux statuts, en 2010. On choisit de consid�rer des distances en termes de salaire puisque c'est finalement l'ad�quation de la trajectoire salariale de l'EIC avec la trajectoire salariale r�elle connue par l'individu de l'Enqu�te Patrimoine qui nous importe. Il est �vident que les distances entre statuts en termes de revenus salariaux et de remplacement ont pu varier entre 1984 et 2010, mais il nous faut d�finir une unique matrice de co�ts de substitution.

\paragraph{Poids relatifs de la distance entre les salaires en 2001 et de la distance entre trajectoires}
Diff�rents essais ont �t� effectu�s pour l'appariement afin de d�terminer l'ordre de grandeur du poids qu'il faut attribuer � la distance entre les trajectoires relativement � celle entre les revenus salariaux et de remplacement en 2001. Lorsque l'on augmente le poids relatif attribu� � la distance entre trajectoires, m�me consid�rablement, le pourcentage de statuts qui ne correspondent pas pour une p�riode donn�e entre ce qui est d�clar� dans l'Enqu�te Patrimoine et ce qui est connu � travers l'EIC ne diminue pas beaucoup. Au contraire, la distance moyenne entre les revenus dans l'Enqu�te Patrimoine et dans l'EIC augmente fortement. On choisit donc d'accorder un poids relativement �lev� � la distance entre les revenus en 2001, puisque cela ne d�t�riore pas grandement la qualit� de l'appariement en termes de trajectoires de statuts, tout en am�liorant assez nettement celle en termes de revenus salariaux et de remplacement en 2001.


\subsection{Trajectoires salariales}

\subsubsection{Imputation des ann�es manquantes}

\paragraph{Imputation des ann�es manquantes pour la Fonction Publique d'Etat}
Les ann�es 1979, 1981 et 1987 sont manquantes en totalit� pour la Fonction Publique d'Etat. Dans le fichier EIC, les salaires de ces ann�es ont �t� imput�es en �tudiant la probabilit� d'�tre entr� dans la fonction publique l'ann�e N plut�t que l'ann�e $N+1$ (si l'individu n'�tait pas d�j� dans la base l'ann�e $N-1$). Cette imputation de la pr�sence dans la fonction publique est conserv�e, mais plut�t que de reprendre simplement le salaire de l'ann�e pr�c�dente, on fait en sorte qu'il y ait une augmentation progressive entre les salaires des ann�es $N-1$, $N$ et $N+1$. Plus pr�cis�ment, on veut qu'en moyenne :\\

$W_{N+1} =  W_{N} \ . \ g_{N+1} \ . \ t$

$W_{N}    =  W_{N-1} \ . \ g_{N} \ . \ t$\\

avec $g$ la croissance nominale des salaires dans le priv� et semi-public (donn�es INSEE) et $t$ un facteur de croissance sp�cifique � la fonction publique et constant entre $N-1$ et $N+1$, et que l'on calcule donc comme $t = \sqrt{\frac{W_{N+1}}{g_{N+1} \ . \ W_{N-1} \ . \ g_{N}}}$.

On choisit ensuite d'appliquer ce facteur (multipli� par la croissance nominale) au salaire de l'ann�e $N-1$ lorsque celui-ci existe, et au salaire de l'ann�e $N+1$ dans le cas contraire, et ce si l'individu est pr�sent dans la fonction publique l'ann�e $N$.
Les salaires des ann�es 1976 et 1977 sont connus dans les DADS mais pas dans les EIC pour la Fonction Publique d'Etat (l'appartenance au fichier de paie des agents de l'Etat est par contre connue). On impute pour l'ann�e 1977 le salaire de 1978 d�flat� par la croissance r�elle des salaires dans le priv� et le semi-public si l'individu appartient � la Fonction Publique cette ann�e-l�, et de m�me pour 1976 en utilisant l'ann�e 1977.


\paragraph{Imputation des ann�es manquantes pour le secteur priv�}
Les ann�es 1981, 1983 et 1990 sont enti�rement manquantes dans les DADS. On estime les revenus salariaux pour chacune de ces ann�es gr�ce � des mod�les tobit.
Pour l'ann�e 1981, on estime tout d'abord un mod�le tobit en r�gressant les salaires nets totaux (issus de l'emploi priv�) pour l'ann�e 1980 sur les salaires (issus du priv�) de 1979 et 1978, l'�ge et l'�ge au carr� en 1980, le sexe, et la pr�sence dans une caisse du secteur priv� en 1980 (connue gr�ce au fichier EIC), ce gr�ce � la proc�dure de Heckman. La s�lection se fait sur les m�mes variables, avec en suppl�ment le statut vis-�-vis de l'emploi en 1978 et 1979. Pour cette estimation, on utilise uniquement les individus pour lesquels il existe au moins un salaire strictement positif pour les ann�es 1978, 1979 et 1982.
Le salaire comme variable explicative est renseign� de fa�on cat�gorielle, par l'appartenance � l'un des neuf premiers d�ciles ou aux cat�gories P90-95, P95-99, P99-99.5, P99.5-99.9, P99.9-99.95, P99.95-99.99, et P99.99-100. Le fait de prendre en compte de fa�on fine l'appartenance au haut de la distribution les ann�es pass�es permet par la suite de simuler des salaires suffisamment �lev�s pour l'ann�e en cours. La variable de statut vis-�-vis de l'emploi est celle qui a �t� construite pour l'appariement, et qui reprend une partie des cat�gories de la variable CYACT de l'enqu�te Patrimoine.
Une fois cette estimation effectu�e, on effectue la pr�diction des salaires nets de l'ann�e 1981 ainsi que de la probabilit� d'avoir un salaire positif, � partir des salaires des ann�es 1980 et 1979, de l'�ge et de l'�ge au carr� en 1981, etc. On n'applique cela qu'aux individus pour lesquels au moins l'un des salaires de 1979, 1980 ou 1982 est strictement positif. On impute un salaire nul aux individus dont les salaires de ces ann�es sont tous nuls.

Pour l'ann�e 1990, on transpose la m�me proc�dure en estimant tout d'abord un mod�le tobit r�gressant les salaires de 1989 sur les donn�es de 1987, 1988 et 1991 (et 1989 pour les variables d'�ge et d'appartenance � une caisse du secteur priv�), puis en effectuant la pr�diction des salaires de 1990 � partir des donn�es concernant les ann�es 1988, 1989 et 1992.
On applique la m�me proc�dure pour l'ann�e 1983. Il n'est pas possible d'utiliser l'ann�e 1981 pour l'estimation et/ou la pr�diction, cette ann�e ayant un statut particulier puisqu'elle a elle-m�me �t� imput�e. On estime ainsi l'effet des salaires etc. des ann�es 1980, 1985 et 1986 sur l'ann�e 1984, puis on obtient la pr�diction des salaires de 1983 en utilisant les donn�es de 1979, 1984 et 1985.


\subsection{R�sultats pr�liminaires : description des trajectoires salariales}

\subsubsection{Volatilit� des revenus sur le cycle de vie}

On consid�re ici les trajectoires salariales sur la p�riode allant de 1976 � 2001, dans le but d'examiner les in�galit�s intra-individuelles de revenu sur une partie du cycle de vie. Pour cela, on s'inspire du travail effectu� par Bj�rklund et Palme (1997) dans le cas de la Su�de. Les auteurs utilisent des mesures d'entropie g�n�ralis�es permettant de d�composer les in�galit�s en deux fractions, in�galit�s interindividuelles de revenu total calcul� sur le cycle de vie d'une part, in�galit�s intra-individuelles au cours du cycle de vie d'autre part. Cette mesure des in�galit�s d�pend d'un param�tre d'aversion � la pauvret� : lorsque celui-ci vaut 0 et 1 respectivement, on retrouve l'indice de Theil-L $I_{0}$ et l'indice de Theil $I_{1}$ respectivement. Le premier correspond � un degr� plus grand d'aversion � la pauvret�, accordant une importance relativement plus grande aux revenus tr�s faibles pour une p�riode donn�e. Les formules correspondant � ces deux indices sont les suivantes :\\

$I_{1}=\frac{1}{n}\sum_{i=1}^{n}\frac{y_{i}}{\overline{y}}\text{log}(\frac{y_{i}}{\overline{y}})$\\

$I_{0}=\frac{1}{n}\sum_{i=1}^{n}\text{log}(\frac{\overline{y}}{y_{i}})$\\


avec $y_{i}$ le revenu de l'individu � la p�riode $i$ et $\overline{y}$ son revenu moyen sur le cycle de vie. $I_{0}$ ne peut �tre calcul� que dans le cas o� les revenus sont non-nuls pour tous les individus � toutes les p�riodes. Cela n'est pas le cas pour nos trajectoires salariales. L'indice de Theil $I_{1}$ au contraire peut �tre calcul� m�me lorsque le revenu des individus est nul � certaines p�riodes, en ne prenant pas en compte ces p�riodes dans la somme (on a $ \lim\limits_{y \to 0} y \ \text{log}(y) =0$). On calcule alors un indice de Theil intra-individuel pour chaque individu, sur la p�riode 1976 � 2001 et pour les g�n�rations 1942 � 1958 (la g�n�ration 1958 a entre 18 et 43 ans et la g�n�ration 1942 entre 34 et 59 ans sur la p�riode consid�r�e). Cet indice individuel renseigne sur la volatilit� des revenus sur le cycle de vie.
Bj�rklund et Palme consid�rent quant � eux les revenus  des individus sur une p�riode de 18 ann�es allant de 1974 � 1991, pour une cat�gorie d'individus " �g�s " (ayant entre 33 et 47 ans en 1974) et une cat�gorie d'individus " jeunes " (ayant entre 18 et 32 ans en 1974). Pour chacun de ces deux groupes, ils examinent la corr�lation entre l'indice de Theil intra-individuel (sur le cycle de vie) et le revenu total sur la p�riode consid�r�e. Celle-ci est significativement n�gative, signifiant que les carri�res salariales sont plus instables dans le bas de la distribution. Lorsqu'ils effectuent la m�me analyse quartile par quartile, les auteurs trouvent une corr�lation significativement n�gative pour le premier quartile, mais non significativement diff�rente de z�ro pour les trois quartiles de revenus (sur le cycle de vie) les plus �lev�s.
Toutes g�n�rations confondues (1942 � 1958), on trouve pour les donn�es EIC-DADS une corr�lation entre indice de Theil intra-individuel et revenus salariaux sur le cycle de vie significativement n�gative (au seuil de 1 \%), d'une valeur de -0.48. Lorsqu'on divise la population en quartiles, les r�sultats sont les suivants :\\

\begin{tabular}{|c|c|c|c|c|}
\hline
           &   P0-25 &  P25-50 &  P50-75 & P75-100 \\
\hline
Indice de Theil moyen &       1.95 &       0.67 &       0.28 &       0.23 \\
\hline
Corr�lation avec le revenu total & -0.74 (***) & -0.47 (***) & -0.18 (***) & 0.18 (***) \\
\hline
\end{tabular}\\

\

Et lorsqu'on examine les individus g�n�ration par g�n�ration, et en d�coupant la distribution � un niveau plus fin (particuli�rement pour le haut de celle-ci), on peut figurer graphiquement les indices de Theil moyens par quantile :
\begin{center}
\includegraphics{Theil_intra.jpg}
\end{center}

\
On observe ainsi une baisse de l'instabilit� des revenus salariaux jusqu'� un certain point, et une remont�e assez prononc�e dans l'extr�mit� haute de la distribution des revenus. Ceci pourrait �tre d� en particulier � la prise en compte des bonus et primes dans les salaires, bonus et primes qui sont variables au cours du temps et pouvant repr�senter des montants consid�rables pour les salari�s les plus r�mun�r�s.


\newpage
\section{Simulation sur le cycle de vie}

voir si �a marche ici

\section{R�sultats pr�liminaires : simulation des allocations familiales (...et de ??)}

\section*{Annexe}

\subsection*{M�thode de matching}

\hspace{-2.5cm}
\begin{tabular}{|p{3.2cm}|p{1.2cm}|p{1.2cm}|p{1.2cm}|p{1.2cm}|p{1.2cm}|p{1.2cm}|p{1.2cm}|p{1.2cm}|p{1.2cm}|p{1.2cm}|}
\hline
           & (1) & (2) & (3) & (4) & (5) &  (6) & (7) & (8) & (9) & (10) \\
\hline
Salari� du public � temps complet (1) &          0 &       8600 &       2900 &      12400 &       5300 &      15120 &      11900 &      19260 &       7100 &      23300 \\
\hline
Salari� du public � temps partiel (2) &       8600 &          0 &       5700 &       3800 &       3300 &       6520 &       3300 &      10660 &       1500 &      14700 \\
\hline
Salari� du priv� � temps complet (3) &       2900 &       5700 &          0 &       9500 &       2400 &      12220 &       9000 &      16360 &       4200 &      20400 \\
\hline
Salari� du priv� � temps partiel (4) &      12400 &       3800 &       9500 &          0 &       7100 &       2720 &        500 &       6860 &       5300 &      10900 \\
\hline
A son compte (5) &       5300 &       3300 &       2400 &       7100 &          0 &       9820 &       6600 &      13960 &       1800 &      18000 \\
\hline
Ch�mage (6) &      15120 &       6520 &      12220 &       2720 &       9820 &          0 &       3220 &       4140 &       8020 &       8180 \\
\hline
Succession de p�riodes d'emploi et de ch�mage (7) &      11900 &       3300 &       9000 &        500 &       6600 &       3220 &          0 &       7360 &       4800 &      11400 \\
\hline
Reprise d'�tudes ou formation (8) &      19260 &      10660 &      16360 &       6860 &      13960 &       4140 &       7360 &          0 &      12160 &       4040 \\
\hline
Pr�retraite (9) &       7100 &       1500 &       4200 &       5300 &       1800 &       8020 &       4800 &      12160 &          0 &      16200 \\
\hline
Inactif et autres (10) &      23300 &      14700 &      20400 &      10900 &      18000 &       8180 &      11400 &       4040 &      16200 &          0 \\
\hline
\end{tabular}

\

Tableau r�sumant la qualit� du matching selon les poids relatifs utilis�s pour les distances ?

\

Tableau d�crivant les salaires imput�s par rapport aux ann�es adjacentes ?

\bibliographystyle{plain}
\bibliography{biblio}



\end{document} 