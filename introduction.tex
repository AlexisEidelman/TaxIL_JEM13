L'objet du projet TaxIPP-Life, débuté en septembre 2012, est la réalisation d'études sur la population française portant sur l'ensemble du cycle de vie. Le regard que l'on porte sur les inégalités peut se voir modifié dans cette optique. Sur des données administratives suédoises, \cite{Bjorklund2011} montre que l'état de pauvreté sur le cycle de vie est temporaire alors que les personnes en haut de la distribution des revenus y restent. Les aller-retours de part et d'autres du seuil de pauvreté est aussi un phénomène connu en France. 
Se pose alors la question de la redistribution sur le cycle de vie. Cette question est en général étudiée sur une base annuelle. Si cette démarche est intéressante pour considérer l'état de la population à un instant donné, elle est souffre de plusieurs limitations. 
D'abord l'estimation de la pauvreté et de la richesse est établie à partir de la situation d'une année seulement et uniquement en fonction des revenus. Ainsi seront pratiquement toujours pauvres les étudiants et souvent le seront les retraités. Si l'on pense que la consommation peut être lissée au cours de la vie et que le revenu permettant d'apréhender cette consommation est le revenu permanent. Une étude annuel ne peut être satisfaisante. Par exemple, un retraité aura en général moins de revenu que lorsqu'il est actif, cependant son niveau de vie a probablement été surestimé lorsqu'il était actif car il a épargné en pensant à sa retraite et est sous-estimé car il peut consommer plus que ses revenus une fois inactif. 
La question du patrimoine est elle aussi primordiale dans l'étude des inégalités. Il est en effet raisonnable de croire qu'un individu ayant un capital (éventuellement un capital attendu à travers un héritage) n'aura pas la même attitude de consommation qu'une personne qui n'en a pas. Si on pense à deux étudiants, l'un aidé par ses parents, l'autre non, dans l'imaginaire collectif celui aidé par ses parents est plus aisé que l'autre. Pourtant, si pour subvenir à ses besoin le second travaille en même temps que ses études, il sera paradoxalement considéré comme plus riche.
 
Le premier point pour justifier une approche sur cycle de vie est donc que l'on peut mieux appréhender la situation des individus. Un deuxième est que si l'on veut étudier la progressivité des transferts ou du système socio-fiscal dans son ensemble, une étude en coupe souffre de limitation importante. 
En effet, les études en coupe doivent contourner la difficulté causée par les transferts dits assuranciels (assurance chômage, retraite et maladie). Ces assurances ne peuvent être vues uniquement comme des transferts. Celui qui cotise ouvre un droit pour plus tard, c'est le principe de l'assurance. Par sa cotisation, il s'assure un revenu futur ou du moins l'espérance d'un revenu. Une étude de la redistribution sur cycle de vie permet d'intégrer ces transferts assuranciels dans l'étude alors qu'ils doivent exclus d'une étude en coupe. Pour ces transferts, comme pour les autres, on est capable d'identifier une redistribution individuel temporelle et une redistribution entre agent. On sépare ainsi la partie d'un prélèvement qu'un individu va \og récupérer \fg \ ou qui lui a été avancée de la partie qui bénéficiera vraiment à d'autres. Pour les prestations, on peut aussi isoler ce qui relève purement de la solidarité de la partie de la prestation que l'individu a lui-même financée \\

L'étandue des questions auxquelles TaxIPP-Life est très grande. On ne les détaillera pas ici car l'état actuel du projet ne permet pas encore d'aborder ces points. Le présent document détaillera essentiellement la méthode utilisée et donnera de premiers résultats provisoires.\\

L'étude sur le cycle de vie exige, et c'est une lapalissade, des données sur l'ensemble de ce cycle de vie. Cela n'est pas sans poser de problème. Le plus évident est que lorsque l'on interroge un individu, il ne connait pas son avenir. Le modèle \til \ contient un module réalisant une projection à un niveau individuel du futur. Ceci sera développé dans la deuxième partie de ce document. La première s'attardera sur un point moins évident qui est que le passé des individus n'est pas non plus si bien connus. En dehors des effets de mémoire qui font qu'une partie du passé est oubliée, aucune enquête ne demande à un individu l'ensemble de ses revenus passés. Des données administratives semblent être la seule solution pour connaitre par exemple la trajectoire professionnelle des individus. Mais les données administratives ne sont pas non plus une panacée, elles ne donnent en général pas d'information sur la vie matrimoniale des personnes, or, savoir qui vit avec qui est déterminant pour étudier la consommation et les niveaux de vie. Enfin, une troisième voie serait l'utilisation d'un panel qui suivrait les individus tout au long de leur vie, toutefois une telle base de données, avec des informations assez riches pour pouvoir appliquer la législation socio-fiscal n'existe pas. Nous avons donc procédé à un matching statistique de données d'enquête et de données administratives. La première partie de document présentera ce matching avec entre les données de l'enquête patrimoine et les données de l'échantillon EIR-EIC.
